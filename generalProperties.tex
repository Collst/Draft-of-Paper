\begin{lem}\label{braidComplex}
Let $G \le S_d$ be a group acting on $\RR^d$ by permuting coordinates. Let $\mathcal{T}_d$ be the partition of $\RR^d$ induced by the braid arrangement $\mathcal{B}_d$. Let $F$ be a face of $\mathcal{T}_d$, and $x$ be in the relative interior of $F$, where $x$ is a lexicographically largest element in its $G$-orbit. Let $y$ be in the relative interior of  $F$ be such that $x \neq y$. Then $y$ is lexicographically maximal in its $G$-orbit.
\end{lem}

\begin{proof}
Assume the opposite. Let $\pi \in G$ be such that $y < \pi \cdot y$, and $\alpha\in[d]$ be the minimal index for which $y_\alpha < y_{\pi^{-1}(\alpha)}$ holds. Therefore, if there is $i\in[d]$ with $1 \le i < \alpha$, then it follows that $y_i = y_{\pi^{-1}(i)}$.

Since $x$ is in the relative interior of $F$, it follows that the indices of the coordinates of $x$ agree with $I_y$ and $E_y$. Therefore, we have that $x_\alpha < x_{\pi^{-1}(\alpha)}$. Furthermore, if $i\in[d]$ with $1 \le i < \alpha$, then it follows that $x_i = x_{\pi^{-1}(i)}$, which implies that $x < \pi \cdot x$. However, $x < \pi \cdot x$ contradicts $x$ being lexicographically maximal in its $G$-orbit.

Therefore, $y$ is lexicographically maximal in its $G$-orbit.
\end{proof}

In light of Lemma \ref{braidComplex}, one might be inclined to collect, for all $x$ in $\RR^d$, the lexicographically maximal elements in their $G$-orbits.

\begin{defn}
Let $K$ be a finite subset of a totally ordered set $T$ (possibly infinite). If $K \neq \emptyset$, then we define the \textbf{minimum} of $K$ to be the smallest element in $K$, that is, \[\min K \equiv \{x \in K : x \le y; y \in K\}.\] Furthermore, the \textbf{maximum} of $K$ is the largest element in $K$, i.e., \[\max K \equiv \{x \in K : x \ge y; y \in K\}.\]
\end{defn}


We define the set of lexicographically largest elements in $\RR^d$ in their respective $G$--orbits as \[\mathcal{F}_{\max} \equiv \{x \in \RR^d : x \ge \pi \cdot x ; \pi \in G\}.\]

In the following proposition, we collect the observations that $\mathcal{F}_{\max}$ is a fundamental domain, and that $\mathcal{F}_{\max}$ is a subcomplex of $\mathcal{T}_d$.

\begin{prop}
Let $G \subseteq S_d$ be a group acting on $\RR^d$. Then $\mathcal{F}_{\max}$ is a subcomplex of $\mathcal{T}_d$. Furthermore, $\mathcal{F}_{\max}$ is a fundamental cell.
\end{prop}

\begin{proof}
It is immediate that $\mathcal{F}_{\max}$ is a fundamental cell. (For any $x\in \RR^d$, we can find the largest element in $G_x$.)

To argue that $\mathcal{F}_{\max}$ is a subcomplex of the braid arrangement, observe that Lemma \ref{braidComplex} tells us that if $F$ is a face of $\mathcal{T}_d$, and $x$ is in both $F$ and $\mathcal{F}_{\max}$, then $F \subseteq \mathcal{F}_{\max}$. Thus, faces of $\mathcal{F}_{\max}$ are the faces of $\mathcal{T}_d$ containing a point that is lexicographically maximal in its $G$-orbit.
\end{proof}

We call $\mathcal{F}_{\max}$ the \textbf{lex--maximal fundamental cell}.

We observe that $\mathcal{F}_{\max}$ is not necessarily a polyhedral complex in general because $\mathcal{F}_{\max}$ need not contain all the points in its topological closure.

\begin{example}
Let \(G = \langle (1\;2)(5\;6),(2\;3)(4\;5),(2\;4)(3\;5)\rangle\) be a subgroup of \(\mathfrak{S}_v\). As a consequence of results in the next section,  \[\mathcal{C} = \{x\in\RR^d : x_1 > x_2; x_2 > x_3; x_2 > x_4; x_2 > x_5; x_1 > x_6\}\] is the interior of \(\mathcal{F}_{\max}\) for $G$. Consider the images of the points below under certain permutations in $G$:
\begin{eqnarray*}
z_1 = (6,6,5,4,3,2) &\xrightarrow{(1 \; 2)(5 \; 6)} &(6,6,5,4,2,3),\\
z_2 = (6,5,5,4,3,2) &\xrightarrow{(2 \; 3)(4 \; 5)} &(6,5,5,3,4,2),\\
z_3 = (6,5,4,5,3,2) &\xrightarrow{(2 \; 4)(3 \; 5)} &(6,5,3,5,4,2),\\
z_4 = (6,5,4,3,5,2) &\xrightarrow{(2 \; 5)(3 \; 4)} &(6,5,3,4,5,2).\\
\end{eqnarray*}
All the $z_i$'s are in $\mathcal{F}_{\max}$. However, even though the points to which the $z_i$'s are mapped are not points in $\mathcal{F}_{\max}$, they are in the topological closure of $\mathcal{C}$, namely \[\overline{\mathcal{C}} = \{x\in\RR^d : x_1 \ge x_2; x_2 \ge x_3; x_2 \ge x_4; x_2 \ge x_5; x_1 \ge x_6\}.\] Therefore, $\mathcal{F}_{\max}$ is not a topologically closed set, so it is not a polyhedral complex.
\end{example}

Instead, \(\mathcal{F}_{\max}\) is an example of a \textbf{partial polyhedral complex}, which is a subset of the faces of a polyhedral complex that is not necessarily closed under taking faces. We will use the term ``complex'' interchangeably with the term ``partial polyhedral complex.'' We view each face of a complex as relatively open.

\begin{example}\label{nonStronglyPureExample}
The subset $\mathcal{P}$ of a triangle $T$ shown in Figure \ref{partialPolyhedralComplex}, which consists of the maximal $2$--dimensional face of $T$ and a vertex, is a $2$--dimensional partial polyhedral complex. The dashed lines mean that $\mathcal{P}$ does not contain any of the edges of $T$.

\vspace{12mm}

\begin{figure}[h!]
\begin{center}
\includegraphics[trim = 0mm 210mm 0mm 30mm, scale = .5]{partialPolyhedralComplex}
\caption{Example of a partial polyhedral complex.}\label{partialPolyhedralComplex}
\end{center}
\end{figure}

\end{example}