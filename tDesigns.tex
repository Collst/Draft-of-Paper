\documentclass[submission%
% if you want to use pdftex and pdflatex doesn't do well for you,
% uncomment the following line
%,pdftex%
% if you have difficulties with hyperref uncomment the following line
%,nohyperref%
% if you have difficulties with fonts uncomment the following line
%,notimes%
]{dmtcs}

% DON'T LOAD ANY STYLES THAT CHANGE THE PAGE LAYOUT
% AND DON'T CHANGE THE PAGE LAYOUT BY HAND, EITHER.


\usepackage[latin1]{inputenc}
\usepackage{subfigure}
\usepackage{bm}
\usepackage{mathtools}
\usepackage{amssymb}

% graphicx is now loaded automatically no need to put this in here anymore.
%
%\usepackage{graphicx}

% just comment this out if you don't have natbib, or if you don't want
% to use it
\usepackage[round]{natbib}

\newcommand{\NN}{\mathbb{N}}
\newcommand{\ZZ}{\mathbb{Z}}
\newcommand{\QQ}{\mathbb{Q}}
\newcommand{\RR}{\mathbb{R}}
\newcommand{\CC}{\mathbb{C}}
\newcommand{\FF}{\mathbb{F}}
\newcommand{\PG}{\text{PG}}
\newcommand{\PGL}{\text{PGL}}
\newcommand{\GL}{\text{GL}}
\newcommand{\PP}{\mathcal{P}}
\newcommand{\conv}{\text{conv}}

\newtheorem{thm}{Theorem}
\newtheorem{lem}[thm]{Lemma}
\newtheorem{prop}[thm]{Proposition}
\newtheorem{cor}[thm]{Corollary}
\newtheorem{conj}[thm]{Conjecture}
\newtheorem{example}[thm]{Example}
\newtheorem{defn}[thm]{Definition}

\author{Felix Breuer \addressmark{1}\thanks{Email: \email{felix@fbreuer.de}. Supported by ...}
  \and Steven Collazos \addressmark{2}\thanks{Email: \email{colla054@umn.edu}. Supported by the National Science Foundation Graduate Research Fellowship program.}
  }
\title[Formatting a submission for DMTCS]{On the Polyedral Geometry of \(t\)-Designs}
% put your affiliation here, not your full address. If you like to give
% away your email address, put it in the \thanks as above.
\address{\addressmark{1}Research Institute for Symbolic Computation, Austria\\
  \addressmark{2}School of Mathematics at University of Minnesota, USA}
\keywords{Fundamental domain, linear extension, pure polyhedral complex.}
% don't try to cheat here, we will check the dates!
\received{1998-10-14}
\revised{\today}
\accepted{tomorrow}
\begin{document}
\maketitle
\begin{abstract}
\paragraph{Abstract.}
Let \(v\) be a positive integer such that \(2 < v\) and let \(X = \{1, 2, \dots, v\}\). Denote by \(\binom{X}{k}\) the collection of all possible \(2\)-subsets of \(X\) and fix total order on it. Define an action of the symmetric group on \(v\) letters \(\mathfrak{S}_v\) on \(\binom{X}{2}\) by taking for all \(b \in \binom{X}{2}\), \(\pi \cdot b := \{\pi(i)  : i \in b\}\), where \(\pi \in \mathfrak{S}_v\). Let \(G \cong \mathfrak{S}_v\) be the subgroup induced by this action on \(\binom{X}{2}\). Let \(G\) act on \(\mathbb{R}^{d}\), where \(d:=\binom{v}{2}\), by permuting coordinates. Let \(\mathcal{F}\) be a fundamental domain (for the action of \(G\) on \(\mathbb{R}^d\)) by taking the collection of points in \(\mathbb{R}^d\) that are lexicographically smallest in their \(G\)-orbit.

In this article, we construct a system of linear inequalities such the closure of the resulting convex region contains \(\mathcal{F}\). We accomplish this goal by constructing a poset whose order cone bounds \(\mathcal{F}\). In particular, we show that linear inequalities for the interior can be obtained from this order cone. We also prove that \(\mathcal{F}\) has the property that any \(k\)-face of \(\mathcal{F}\) is a face of a \((k+1)\)-dimensional face of \(\mathcal{F}\).

\paragraph{R\'esum\'e.}


\end{abstract}


\section{Introduction}\label{intro}
\label{sec:in}

The purpose of this article is two-fold. First, we prove that any collection of lexicographically smallest elements in their \(G\)-orbit has a property we refer to as \textit{strong purity}. Second, we exhibit a poset whose order cone bounds a fundamental domain for a certain group.

Let $X = \{i_1, i_2, \dots , i_d\}$, with $i_1 \succ i_2 \succ \dots \succ i_d$, be a totally ordered set with $d$ elements. Let  $G \le S_X$ be a subgroup acting on $\RR^X$ by permuting coordinates. In other words, for any $\pi \in G$, \[\pi \cdot x \equiv (x_{\pi^{-1}(i_1)}, x_{\pi^{-1}(i_2)}, \dots , x_{\pi^{-1}(i_d)}),\] where $x = (x_{i_1}, x_{i_2}, \dots , x_{i_d})\in\RR^X$.

\begin{defn}
Let $\mathcal{F} \subseteq \RR^d$ and $G$ be a finite group acting on $\RR^d$ by permuting coordinates. Suppose that for every $x\in\RR^d$, the equality $\vert x^G \cap \mathcal{F}\vert = 1$ holds, where \(x^G\) denotes the \(G\)-orbit of \(x\).

We call $\mathcal{F}$ a \textbf{fundamental domain} for \(G\).
\end{defn}

An alternative way of defining a fundamental region is as a subset of $\RR^d$ satisfying the following two properties: \begin{enumerate}
\item For every $x\in\mathcal{F}$ and $\pi \in G\backslash\{e\}$, where $e$ denotes the identity of $G$, we have that $\pi \cdot x \notin \mathcal{F}$;
\item For every $x\in\RR^X$, there exists $\pi \in G$ such that $\pi \cdot x \in \mathcal{F}$.
\end{enumerate} %We will refer to the first property as $\boldsymbol G$-\textbf{uniqueness}, and the latter one as $\boldsymbol G$-\textbf{existence}.

We introduce a partition of $\RR^d$ we will use often:

\begin{defn}[\cite{hyperplaneArrg}]
For $i, j \in [d]$, with $i \neq j$, let $H_{ij} = \{x\in\RR^d : x_i - x_j = 0; 1 \le i < j \le d \}$. The \textbf{braid arrangement}, denoted $\mathcal{B}_d$, on $\RR^d$ is the set \[\mathcal{B}_d = \{H_{ij} : 1 \le i < j \le n \}.\]
\end{defn}

The braid arrangement $\mathcal{B}_d$ induces a collection of cones, and their relative interiors form a partition of $\RR^d$ \cite[p. 192]{ziegler}. Let $\mathcal{T}_d$ denote the collection of these regions.

\begin{defn}
Let $F$ be a subset of $\mathcal{T}_d$. For any $z$ in the relative interior of $F$, define \begin{eqnarray*} I_z &\equiv &\{(i,j) \in [d]^2 : i \neq j; z_i - z_j > 0 \},\\ E_z &\equiv &\{(i,j) \in [d]^2 : i \neq j; z_i - z_j = 0\}. \end{eqnarray*} If all pairs of coordinates fall in either $I_z$ or $E_z$, then we call $F$ a \textbf{face} of $\mathcal{T}_d$, in which case we say that $F$ is a face of the braid arrangement.
\end{defn}

This paper is organized as follows. In the next section, we introduce the fundamental domain we wish to study, which we denote \(\mathcal{F}_{\max}\), and prove some of its properties. In Section \ref{purity}, we prove that such a fundamental domain possess a property called strong purity. In Section \ref{system}, we work with a specific fundamental domain \(\mathcal{F}_{\max}\) for which we are able to construct a poset whose order cone yields a system of linear inequalities for the interior of \(\mathcal{F}_{\max}\).




\section{The Lex-Maximal Fundamental Domain}\label{properties}
\label{sec:in}

\begin{lem}\label{braidComplex}
Let $G \le S_d$ be a group acting on $\RR^d$ by permuting coordinates. Let $\mathcal{T}_d$ be the partition of $\RR^d$ induced by the braid arrangement $\mathcal{B}_d$. Let $F$ be a face of $\mathcal{T}_d$, and $x$ be in the relative interior of $F$, where $x$ is a lexicographically largest element in its $G$-orbit. Let $y$ be in the relative interior of  $F$ be such that $x \neq y$. Then $y$ is lexicographically maximal in its $G$-orbit.
\end{lem}

\begin{proof}
Assume the opposite. Let $\pi \in G$ be such that $y < \pi \cdot y$, and $\alpha\in[d]$ be the minimal index for which $y_\alpha < y_{\pi^{-1}(\alpha)}$ holds. Therefore, if there is $i\in[d]$ with $1 \le i < \alpha$, then it follows that $y_i = y_{\pi^{-1}(i)}$.

Since $x$ is in the relative interior of $F$, it follows that the indices of the coordinates of $x$ agree with $I_y$ and $E_y$. Therefore, we have that $x_\alpha < x_{\pi^{-1}(\alpha)}$. Furthermore, if $i\in[d]$ with $1 \le i < \alpha$, then it follows that $x_i = x_{\pi^{-1}(i)}$, which implies that $x < \pi \cdot x$. However, $x < \pi \cdot x$ contradicts $x$ being lexicographically maximal in its $G$-orbit.

Therefore, $y$ is lexicographically maximal in its $G$-orbit.
\end{proof}

In light of Lemma \ref{braidComplex}, one might be inclined to collect, for all $x$ in $\RR^d$, the lexicographically maximal elements in their $G$-orbits.

\begin{defn}
Let $K$ be a finite subset of a totally ordered set $T$ (possibly infinite). If $K \neq \emptyset$, then we define the \textbf{minimum} of $K$ to be the smallest element in $K$, that is, \[\min K \equiv \{x \in K : x \le y; y \in K\}.\] Furthermore, the \textbf{maximum} of $K$ is the largest element in $K$, i.e., \[\max K \equiv \{x \in K : x \ge y; y \in K\}.\]
\end{defn}


We define the set of lexicographically largest elements in $\RR^d$ in their respective $G$--orbits as \[\mathcal{F}_{\max} \equiv \{x \in \RR^d : x \ge \pi \cdot x ; \pi \in G\}.\]

In the following proposition, we collect the observations that $\mathcal{F}_{\max}$ is a fundamental domain, and that $\mathcal{F}_{\max}$ is a subcomplex of $\mathcal{T}_d$.

\begin{prop}
Let $G \subseteq S_d$ be a group acting on $\RR^d$. Then $\mathcal{F}_{\max}$ is a subcomplex of $\mathcal{T}_d$. Furthermore, $\mathcal{F}_{\max}$ is a fundamental cell.
\end{prop}

\begin{proof}
It is immediate that $\mathcal{F}_{\max}$ is a fundamental cell. (For any $x\in \RR^d$, we can find the largest element in $G_x$.)

To argue that $\mathcal{F}_{\max}$ is a subcomplex of the braid arrangement, observe that Lemma \ref{braidComplex} tells us that if $F$ is a face of $\mathcal{T}_d$, and $x$ is in both $F$ and $\mathcal{F}_{\max}$, then $F \subseteq \mathcal{F}_{\max}$. Thus, faces of $\mathcal{F}_{\max}$ are the faces of $\mathcal{T}_d$ containing a point that is lexicographically maximal in its $G$-orbit.
\end{proof}

We call $\mathcal{F}_{\max}$ the \textbf{lex--maximal fundamental cell}.

We observe that $\mathcal{F}_{\max}$ is not necessarily a polyhedral complex in general because $\mathcal{F}_{\max}$ need not contain all the points in its topological closure.

\begin{example}
Let \(G = \langle (1\;2)(5\;6),(2\;3)(4\;5),(2\;4)(3\;5)\rangle\) be a subgroup of \(\mathfrak{S}_v\). As a consequence of results in the next section,  \[\mathcal{C} = \{x\in\RR^d : x_1 > x_2; x_2 > x_3; x_2 > x_4; x_2 > x_5; x_1 > x_6\}\] is the interior of \(\mathcal{F}_{\max}\) for $G$. Consider the images of the points below under certain permutations in $G$:
\begin{eqnarray*}
z_1 = (6,6,5,4,3,2) &\xrightarrow{(1 \; 2)(5 \; 6)} &(6,6,5,4,2,3),\\
z_2 = (6,5,5,4,3,2) &\xrightarrow{(2 \; 3)(4 \; 5)} &(6,5,5,3,4,2),\\
z_3 = (6,5,4,5,3,2) &\xrightarrow{(2 \; 4)(3 \; 5)} &(6,5,3,5,4,2),\\
z_4 = (6,5,4,3,5,2) &\xrightarrow{(2 \; 5)(3 \; 4)} &(6,5,3,4,5,2).\\
\end{eqnarray*}
All the $z_i$'s are in $\mathcal{F}_{\max}$. However, even though the points to which the $z_i$'s are mapped are not points in $\mathcal{F}_{\max}$, they are in the topological closure of $\mathcal{C}$, namely \[\overline{\mathcal{C}} = \{x\in\RR^d : x_1 \ge x_2; x_2 \ge x_3; x_2 \ge x_4; x_2 \ge x_5; x_1 \ge x_6\}.\] Therefore, $\mathcal{F}_{\max}$ is not a topologically closed set, so it is not a polyhedral complex.
\end{example}

Instead, \(\mathcal{F}_{\max}\) is an example of a \textbf{partial polyhedral complex}, which is a subset of the faces of a polyhedral complex that is not necessarily closed under taking faces. We will use the term ``complex'' interchangeably with the term ``partial polyhedral complex.'' We view each face of a complex as relatively open.

\begin{example}\label{nonStronglyPureExample}
The subset $\mathcal{P}$ of a triangle $T$ shown in Figure \ref{partialPolyhedralComplex}, which consists of the maximal $2$--dimensional face of $T$ and a vertex, is a $2$--dimensional partial polyhedral complex. The dashed lines mean that $\mathcal{P}$ does not contain any of the edges of $T$.

\vspace{12mm}

\begin{figure}[h!]
\begin{center}
\includegraphics[trim = 0mm 210mm 0mm 30mm, scale = .5]{partialPolyhedralComplex}
\caption{Example of a partial polyhedral complex.}\label{partialPolyhedralComplex}
\end{center}
\end{figure}

\end{example}


\section{Strong Purity}\label{purity}

We will now present one of the main results of the manuscript. Before doing so, we need the following notion.

\begin{defn}
We say that a $d$--dimensional partial polyhedral complex $\mathcal{P}$ in $\RR^d$ is \textbf{strongly pure} if for any $k$-dimensional face $\sigma$ of $\mathcal{P}$, where $0 \le k < d$, we have that $\sigma$ is a face of a $(k + 1)$-dimensional face of $\mathcal{P}$.
\end{defn}

While the notions of purity and strong purity are the same in polyhedra, they need not agree for partial polyhedral complexes. In Example \ref{nonStronglyPureExample}, for instance, we have that $\mathcal{P}$ is pure because the vertex (the only non--maximal face in $\mathcal{P}$) is a face of the relative interior of the triangle. However, $\mathcal{P}$ is not strongly pure because a vertex is a $0$--dimensional face of $T$, whereas the relative interior of $T$ is $2$--dimensional.

\begin{thm}\label{stronglyPure}
Let $G$ be a subgroup of \(\mathfrak{S}_v\) acting on $\RR^d$ by permuting coordinates. Then the boundary of $\mathcal{F}_{\max}$ is strongly pure.
\end{thm}

\begin{proof}
Before providing the proof for Theorem \ref{stronglyPure}, we give an outline. We will pick an arbitrary point $x$ in a $k$-dimensional face $\sigma$ of $\mathcal{F}_{\max}$, where $1 \le k < d$. Since the dimension of $\sigma$ is less than $d$, we know that at least one pair of coordinates of $x$ must be equal. Among all the possible coordinates that are equal, we choose a suitable index $I$ and add a small enough number $\epsilon_n(I)$, where $\epsilon_n(I) \rightarrow 0$ as $n \rightarrow \infty$. We will show by contradiction that the terms of the resulting sequence $x^{(n)}$ are lexicographically maximal in their $G$--orbit. Since the $x^{(n)}$ are in a $(k + 1)$--dimensional face $\sigma^\prime$, and $x^{(n)}$ converges to $x$, it follows that $\sigma^\prime$ has $\sigma$ as a face.

Let $\sigma$ be a $k$-dimensional face of $\mathcal{F}_{\max}$, where $0 \le k < d$, and $x = (x_1, x_2, \dots , x_d)$ be in the relative interior of $\sigma$.

We introduce a multiset consisting of coordinates of $x$. Specifically, define \[K \equiv \bigcup_{\substack{i,j \in P  \\ i \neq j}} \{x_i : x_i = x_j\}.\] We know that $K \neq \emptyset$ because $\sigma$ is a non-trivial face of $\mathcal{F}_{\max}$ ---so there is at least one pair of coordinates of $x$ that are equal.

Let $I \equiv \min\{i \in [d] : x_i = \max K\}$. Define $r \equiv \min S$, where \[S \equiv \{\vert x_i - x_j \vert : i \neq j; x_i \neq x_j\},\] and let $N\in\NN$ be such that $\frac{d}{N} < r$ when $S \neq \emptyset$. (If $S = \emptyset$, then all the coordinates of $x$ are equal. Therefore, the sequence $x^{(n)} = (x_1 + \frac{1}{n}, x_2, \dots , x_d)$ has all the desired properties. First, it is clear that $x^{(n)}$ is lexicographically maximal in its $G$-orbit. For every $n\in\NN$, we have that $x^{(n)}$ is a point in a $(k+1)$-dimensional face. Further, $x^{(n)}$ converges to $x$ as $n \rightarrow \infty$.)

We claim that for every $n\in\NN$, the sequence \[x^{(n)} = (x_1 + \epsilon_n(1), x_2 + \epsilon_n(2), \dots , x_d + \epsilon_n(d)),\] where $\epsilon_n(i) = \frac{i}{N + n}$ if $i = I$, and $\epsilon_n(i) = 0$ otherwise, is lexicographically maximal in its $G$-orbit. Observe that $(x^{(n)})_i = x_i$ as long as $i \neq I$. If $i = I$, then $(x^{(n)})_i > x_i$. As an additional observation, note for any $n\in\NN$, we have that $(x^{(n)})_I \neq (x^{(n)})_i$ for any $i \neq I$.

We will prove by contradiction that $x^{(n)} \in \mathcal{F}_{\max}$. Let $\pi \in G$ and $n\in\NN$ be such that $x^{(n)} < \pi \cdot x^{(n)}$. Let $\alpha \in [d]$ be the minimal index with the property that $(x^{(n)})_\alpha < (x^{(n)})_{\pi^{-1}(\alpha)}$. Note that if $i$ is such that $1 \le i < \alpha$, then $(x^{(n)})_i = (x^{(n)})_{\pi^{-1}(i)}$. We argue that this setup implies that $x < \pi \cdot x$.

We have two cases:

\begin{enumerate}
\item Suppose that $\alpha = I$. Then \[(x^{(n)})_I < (x^{(n)})_{\pi^{-1}(I)} \; \Longrightarrow \; x_I + \frac{I}{N + n} < x_{\pi^{-1}(I)}.\] To see why the second inequality is true, note that if $\alpha$ were a fixed point of $\pi^{-1}$, then $(x^{(n)})_\alpha < (x^{(n)})_{\pi^{-1}(\alpha)}$, contradicting $(x^{(n)})_\alpha < (x^{(n)})_{\pi^{-1}(\alpha)}$. Thus, $\pi^{-1}(I) \neq I$, implying that $x_I + \frac{I}{N + n} < x_{\pi^{-1}(I)}$.  From this last inequality, we have that $x_I < x_{\pi^{-1}(I)}$ because if $x_I = x_{\pi^{-1}(I)}$, then it follows that $x_I + \frac{I}{N + n}$ is greater than $x_{\pi^{-1}(I)}$, which is inconsistent with the inequality $(x^{(n)})_I < (x^{(n)})_{\pi^{-1}(I)}$.

    On the other hand, if there is $i\in[d]$ with $1 \le i < I$, then the equality $x_i = (x^{(n)})_i$ holds because $i\neq I$. Furthermore, since $i < \alpha$, it follows that $(x^{(n)})_i = (x^{(n)})_{\pi^{-1}(i)}$. To summarize, \[x_i = (x^{(n)})_i = (x^{(n)})_{\pi^{-1}(i)} = x_{\pi^{-1}(i)}.\]

    Hence, $x < \pi \cdot x$.
\item Now assume that $\alpha \neq I$. Note that we only need to consider the case when $\alpha < I$ because otherwise,  \[(x^{(n)})_I = (x^{(n)})_{\pi^{-1}(I)} \; \Longrightarrow \; x_{\pi^{-1}(I)} = x_I + \frac{I}{N + n},\] which is impossible due to how we chose $N$.

    Since $\alpha \neq I$, we know that $(x^{(n)})_\alpha = x_\alpha$, so $x_\alpha < (x^{(n)})_{\pi^{-1}(\alpha)}$. Now we consider two separate subcases:
    \begin{itemize}
    \item Suppose $\pi^{-1}(\alpha) = I$. By assumption, $(x^{(n)})_\alpha  <  (x^{(n)})_{\pi^{-1}(\alpha)}$, which implies \[x_\alpha < x_I + \frac{I}{N + n} \; \Longrightarrow \; x_\alpha < x_{\pi^{-1}(\alpha)},\] where the second step follows by our definition of $N$. For any $i$ with $1 \le i < \alpha$, we have \[x_i = (x^{(n)})_i = (x^{(n)})_{\pi^{-1}(i)} = x_{\pi^{-1}(i)},\] because $\pi^{-1}(i) \neq I$. Thus, $x < \pi \cdot x$.
    \item Assume now that $\pi^{-1}(\alpha) \neq I$. Since $x_\alpha < (x^{(n)})_{\pi^{-1}(\alpha)}$, it follows that $x_\alpha < x_{\pi^{-1}(\alpha)}$. For any $i\in[d]$ with $1 \le i < \alpha$, we see that \[x_i = (x^{(n)})_i = (x^{(n)})_{\pi^{-1}(i)} = x_{\pi^{-1}(i)},\] where the last equality is obtained by observing that if $\pi^{-1}(i) = I$, then $x_i = x_I + \frac{I}{N + n}$, which is not possible. Since $x_i = x_{\pi^{-1}(i)}$ for all $i < \alpha$, we infer that $x < \pi\cdot x$.
    \end{itemize}
    Therefore, in either case we reach a contradiction.
\end{enumerate}

Since the $x^{(n)}$ are in the relative interior of the same face, say $\sigma^\prime$, we conclude that for all $n\in\NN$, we have that $x^{(n)}$ is lexicographically maximal in its $G$--orbit by Lemma \ref{braidComplex}. Thus, $\sigma$ is a face of $\sigma^\prime$, so $\mathcal{F}_{\max}$ is strongly pure.
\end{proof}



\section{Construction of System of Linear Inequalities Bounding a Specific Fundamental Domain}\label{system}

First, we introduce a poset that will enable us to exhibit an inequality description for the interior of the lex--maximal fundamental cell $\mathcal{F}_{\max}$. 

Before moving further, we introduce the following group: Let \(X = \{1, 2, \dots, v\}\), where \(v > 2\), and let \(\binom{X}{2}\) be a totally ordered set on the collection of 2-subsets of \(X\). Define a group action on \(\binom{X}{2}\) by \(\mathfrak{S}_v\) via \[\pi \cdot b := \{\pi(i) : i \in b\},\] where \(\pi \in \mathfrak{S}_v\). Note such a group action will induce a group \(G \cong \mathfrak{S}_v\). 

\begin{example}
Let \(X = \{1, 2, 3, 4\}\) and 
\[
\binom{X}{2} = \left(12, 13, 14, 23, 24, 34\right),
\]
where \(ij\) is shorthand notation for \(\{i,j\}\). Then for \(\pi := (1\;2) \in \mathfrak{S}_4\),
\begin{eqnarray*}
\pi \cdot \binom{X}{2} &= &\left(\pi\cdot12, \pi\cdot13, \pi\cdot14, \pi\cdot23, \pi\cdot24, \pi\cdot34\right)\\
&= &\left(12, 23, 24, 13, 14, 34\right).
\end{eqnarray*}
Therefore, \((1\;2)\) induces the permutation \((2\;4)(3\;5) \in G\).
\end{example}

Now that we have introduced the group whose lex-maximal fundamental we propose to analyze, we define the following poset:

\begin{defn}\label{interiorPoset}
Let $X = \{1, 2, \dots , v\}$, and $d = \binom{v}{2}$. Let $A_1$ be the antichain consisting of all $2$--subsets of $X$ that do not contain $1$ nor $2$; $A_2$ be the antichain consisting of all $2$--subsets of $X$ that contain $2$, but do not contain $1$. The collection $B$ consists of those $2$--subsets containing a $1$ and it is ordered by the reverse lexicographical order. Define $P$ to be the poset on $2$--subsets of $X$ with the following relations:

\begin{itemize}
\item For any $Z \in \binom{X}{2}$, we have $\{1,2\} \succeq Z$;
\item For all $Y \in A_1$, the relation $\{1,3\} \succ Y$ holds;
\item For $\{1,i\}, \{1,j\} \in B$, we have that $\{1,i\} \succ \{1,j\}$ when $i < j$.
\end{itemize}

We call $P$ the \textbf{interior poset}. (Refer to Figure \ref{Hasse2}.)
\end{defn}

\begin{figure}[h!]
\begin{center}
\includegraphics[trim = 0mm 190mm 0mm 30mm, scale = .5]{HasseDiagramNchoose2}
\caption{Hasse diagram of $P$ in Definition \ref{interiorPoset}.}\label{Hasse2}
\end{center}
\end{figure}

For the remainder of the paper, we will reserve the letter $P$ to refer to an interior poset.

Here is a notion that will link interior posets with fundamental domains:
\begin{defn}[\cite{orderCone}]
Suppose $P$ is an interior poset with $d$ elements. The \textbf{order cone} of $P$, denoted $\mathcal{C}(P)$, is the set \[\mathcal{C}(P) \equiv \bigcap_{\substack{i,j \in P  \\ i \gtrdot j}} \{ x\in\RR^P : x_i \ge x_j \},\] where $i \gtrdot j$ means that $i$ covers $j$. Equivalently, $\mathcal{C}(P)$ is the set of all order-preserving maps $f : P \rightarrow \RR$, where we identify every possible $f$ with points $x = (x_1, x_2, \dots , x_d)$ in $\RR^P$ such that $x_i > x_j$ if and only if $f(i) > f(j)$.
\end{defn}

The next two results are intended to show why working with $P$ is helpful in constructing a fundamental cell. First, we introduce a definition.

\begin{defn}
Let $P$ be an interior poset and $\pi \in S_X$. We say that $\pi$ is \textbf{order-preserving with respect to} $\boldsymbol P$ if for every $\alpha, \beta \in P$ with $\alpha \succ \beta$, we have that either $\pi \cdot \alpha \succ \pi \cdot \beta$, or $\{\pi \cdot \alpha, \pi \cdot \beta\}$ is an antichain.
\end{defn}

\iffalse
\begin{defn}
Let $\pi \in S_X$ and $\langle \pi \rangle$ be the cyclic group generated by $\pi$. We say that $\pi$ is \textbf{free with respect to} $\boldsymbol P$ if for every $i \in P$, the orbit $\langle \pi \rangle_i$ is an antichain.
\end{defn}
\fi

\begin{prop}\label{noOrderPreservingPerms}
Let $P$ be an interior poset. Then $S_X\setminus\{e\}$ has no order-preserving permutations with respect to $P$.
\end{prop}

\begin{proof}
Suppose $\pi(1) = 2$ and $\pi(2) = 1$, or $\pi(1) = 1$ and $\pi(2) = 2$. (Otherwise, $\pi \cdot \{1, 2\} \neq \{1, 2\}$, so $\pi$ would not be order-preserving.)

\begin{enumerate}
\item Note that $\pi \cdot \{1, 3\} = \{2, \pi(3)\}$ and $\pi \cdot \{2, i\} = \{1, \pi(i)\}$. Since $\pi(3) \neq 1$, then $\pi\cdot\{1, 3\} \in B$. There exists $i \in [v]$ such that $\pi(i) = 3$, then $\pi \cdot \{2, i\} \succ \pi \cdot \{1, 3\}$, which is a contradiction.
\item Let $i > j$ such that $\pi(i) < \pi(j)$. Although $\{1, i\} \prec \{1, j\}$, we have that $\pi \cdot \{1, i\} \not \prec \pi \cdot \{1, j\}$.
\end{enumerate}

Thus, no $\pi$ in $S_X\setminus\{e\}$ is order-preserving.
\end{proof}

\begin{prop}\label{noOrderPreservingIsGUniqueness}
For every $\pi \in S_X\setminus\{e\}$, $\pi$ is not order-preserving with respect to $P$ if and only if $\mathcal{C}(P)^\circ$ has the $G$-uniqueness property.
\end{prop}

\begin{proof}
\textit{Forward implication:} Let $\pi \in S_X$ be a permutation that does not preserve the order in $P$, and let $x\in\mathcal{C}(P)$. Let $i,j\in P$ be such that $i \succ j$, and $\pi\cdot i \succ \pi\cdot j$. If $i\succ j$, then $x_i > x_j$, so for $\pi\cdot x$, we have that $x_{\pi^{-1}(i)} < x_{\pi^{-1}(j)}$. Thus, $\pi\cdot x$ is not in $\mathcal{C}(P)$.


\textit{Backward implication:} We prove the contrapositive. Take the longest chain in $P$, and take $\alpha \gtrdot \beta$ such that $\alpha$ and $\beta$ are mapped non-trivially by $\pi \in G$. Since $\pi$ is order-preserving, we have that $\pi\cdot\{\alpha, \beta\}$ is an antichain in $P$, or $\pi\cdot\alpha \succ \pi\cdot\beta$. Hence, if $x \in \mathcal{C}(P)$ satisfies $x_\alpha > x_\beta$, then $x_{\pi^{-1}(\alpha)} > x_{\pi^{-1}(\beta)}$ will be a valid inequality in $\mathcal{C}(P)$, so $\pi\cdot x\in\mathcal{C}(P)$.
\end{proof}


\iffalse
\begin{prop}
If $\pi \in S_X\setminus\{e\}$ is a free permutation, then $\pi$ is order-preserving.
\end{prop}

\begin{proof}
Let $\pi \in S_X\setminus\{e\}$ be free, and let $\alpha, \beta \in P$ be given, with $\alpha \succ \beta$. Assume that $\alpha \neq \{1,2\}$ ---otherwise, $\alpha$ and $\pi\cdot\alpha$ would be comparable, contradicting $\pi$ being a free permutation. Further assume that $\pi$ acts trivially on $\{1,2\}$, for $\alpha$ and $\pi\cdot\alpha$ would be comparable otherwise. Additionally, suppose $\pi$ acts non-trivially on $\alpha$ and $\beta$. (We will deal shortly with the case where $\pi$ acts trivially on either $\alpha$ or $\beta$.)

We will prove that either $\pi\cdot\alpha \succ \pi\cdot\beta$, or $\pi\cdot\alpha$ and $\pi\cdot\beta$ are incomparable. Since $\alpha$ and $\beta$ are comparable, it follows that $\alpha = \{1, j\}$ for some $j\in[d]$. Further, $\pi\cdot\alpha\in A_1$, or $\pi\cdot\alpha\in A_2$ because $\langle \pi \rangle_\alpha$ is an antichain.

\begin{enumerate}
\item If $\pi\cdot\alpha\in A_1$, then $\pi\cdot\beta$ is incomparable with $\pi\cdot\alpha$ because $\pi\cdot\beta \neq \{1,2\}$.
\item If $\pi\cdot\alpha\in A_2$, then either $\pi\cdot\beta\in A_1$ or $\pi\cdot\beta\prec\{1,3\}$. It is easy to see that $\pi\cdot\beta \neq \{1,3\}$: It would follow that $\beta$ and $\pi\cdot\beta$ are comparable, contradicting the freeness of $\pi$.

    Suppose $\pi\cdot\beta\notin A_1$, which implies that $\pi\cdot\beta \in B$, with $\pi\cdot\beta \prec \{1,3\}$, or $\pi\cdot\beta\in A_2$. In either case, $\pi\cdot\alpha$ and $\pi\cdot\beta$ are incomparable.

    Next suppose that $\pi\cdot\beta\not\prec\{1,3\}$, so either $\pi\cdot\beta\in A_2$ or $\pi\cdot\beta\in A_1$. Once more, we see that $\pi\cdot\alpha$ and $\pi\cdot\beta$ are not comparable.
\end{enumerate}

Notice in particular that if $\pi$ acts non-trivially on $\alpha$ and $\beta$, then $\alpha\succ\beta$ requires that $\pi\cdot\alpha$ and $\pi\cdot\beta$ be incomparable.

If we assume that $\pi$ acts trivially on either $\alpha$ or $\beta$ (but not both), then, unless either $\alpha$ or $\beta$ is $\{1,3\}$, we have that $\pi\cdot\alpha$ and $\pi\cdot\beta$ will be incomparable. If $\{1,3\}$ is acted upon non-trivially, then $\pi\cdot\{1,3\} \in A_1$, in which case again $\pi\cdot\alpha$ and $\pi\cdot\beta$ are incomparable.

We conclude that $\pi$ is order-preserving.
\end{proof}
\fi

Given this evidence that studying $P$ could be helpful, we proceed to show that the covering relations in $P$ enable us to describe the interior of $\mathcal{F}_{\max}$ in Theorem \ref{kEqualsTwoIsLexMax}. Specifically, we will show that the interior of $\mathcal{C}(P)$ is the interior of the lex--maximal fundamental cell. To go about the proof of this claim, we establish a result concerning linear extensions of $P$.

\begin{thm}\label{nChoose2Case}
Let $P$ be an interior poset, and $X$ a $v$--set. Let $v, k \in \NN$, where $v < k$. Define $\mathrm{Perm}\binom{X}{k}$ to be the set of all permutations on $\binom{X}{k}$. Let \[G \backslash \backslash \mathrm{Perm}\binom{X}{k} \equiv \left\{\mathcal{O}_{\mathfrak{S}} : \mathfrak{S}\in\mathrm{Perm}\binom{X}{k}\right\},\] that is, $G\backslash\backslash\mathrm{Perm}\binom{X}{k}$ is the set of all $G$--orbits of permutations on $\binom{X}{k}$. Then there exists a unique permutation in every $G$--orbit $\mathcal{O}_{\mathfrak{S}} \in G \backslash \backslash \mathrm{Perm}\binom{X}{k}$ that is a linear extension of $P$.
\end{thm}

\begin{proof}
\textit{Existence:} Let $T = (T_1, T_2, \dots , T_d)$ be a permutation on $\binom{X}{k}$. We will construct a $\pi\in G$ such that $\pi\cdot T$ is a linear extension of $P$.

\begin{enumerate}
\item If $T_1 \neq \{1,2\}$, then let $p\in S_X$ be such that $p\cdot T_1 = \{1,2\}$.
\item Let $\omega \equiv \min\{i\in[d]\setminus\{1\} : 1\in T_i\}$ and $\tau \equiv \min\{i\in[d]\setminus\{2\} : 2 \in T_i\}$. There are two cases to consider:
    \begin{itemize}
    \item $\mathbf{\omega < \tau:}$ Then let $q = (j\;3)$, so that $q \cdot T_\omega = \{1, 3\}$, where $j\in T_\omega$.
    \item $\mathbf{\tau < \omega:}$ Then let $q = (1 \; 2)(3 \; h)$, so that $q \cdot T_\tau = \{1, 3\}$, where $h \in T_\tau$.
    \end{itemize}
Notice that in either case, the resulting permutation $q \cdot T$ has the property that $\{1,2\}$ is the minimum, and $\{1,3\} > S$, where $S\in A_2$.

\item If $\{\{1, 2\}, \{1, 3\}, \dots , \{1, v\} \}$ is a subposet of $T$, then we are done. Otherwise, we can find a permutation $r\in S_X$ such that $r\cdot i_1 = \{1,2\} \prec r \cdot i_2 \prec \dots \prec r \cdot i_v = \{1, v\}$, where $i_j \in B$.
\end{enumerate}

We conclude that $(rqp) \cdot T$ is a linear extension of $P$.

\textit{Uniqueness:} Let $L$ be a linear extension of $P$. Let $p \in S_X \setminus \{e\}$, where $e$ denotes the identity element in $S_X$. Without loss of generality, assume $p \cdot \{1, 2\} = \{1, 2\}$. (Otherwise, the resulting permutation $p\cdot L$ no longer has $\{1, 2\}$ in the first coordinate, so $p\cdot L$ cannot possibly be a linear extension of $P$.) We have two cases to consider:

\begin{enumerate}
\item Suppose $p(1) = 1$ and $p(2) = 2$. Since $L$ is a linear extension of $P$, then $\{1, 2\} \succ \{1, 3\} \succ \dots \succ \{1, v\}$ is a subposet of $L$. Since $p$ is a non-trivial permutation, $p$ cannot respect the ordering on this subposet, so $p\cdot L$ is not a linear extension of $P$.
\item Suppose $p$ has the cycle $(1 \; 2)$. Observe that $\pi \cdot \{1, 3 \} = \{2, \pi(3)\}$ and $\pi \cdot \{2, j\} = \{1, 3\}$, for some $j\in\{2, 3, \dots, v-1\}$. Since $\{1,3\} \succ \{2, j\}$ is a covering relation in $P$, it follows that \[\pi \cdot \{1, 3\} = \{2, \pi(3)\} \not\succ \pi\cdot \{2, j\} = \{1, 3\}\] because $\pi(3) \neq 1$.
\end{enumerate}

We conclude that for any $\pi\in S_X \setminus\{e\}$, we have that $\pi \cdot L$ is not a linear extension of $P$.
\end{proof}

Remark: Let $Y = \mathrm{Perm}\binom{X}{2}$. Notice that by Burnside's lemma \cite{dummit}, \[\vert G\backslash\backslash Y\vert = \frac{1}{G}\sum_{\pi \in G} \vert \mathrm{Fix}(\pi) \vert = \frac{\binom{v}{2}!}{v!},\] where $\mathrm{Fix}(\pi)$ denotes the set of elements in $Y$ fixed by $\pi$. Therefore, $[S_d : G]$ is the number of linear extensions of $P$.

Now we can prove that $\mathcal{C}(P)$ contains all the points we need.

\begin{prop}\label{representativesInClosure}
The order cone $\mathcal{C}(P)$ has the $G$-existence property.
\end{prop}

\begin{proof}
Let $x$ be a point in $\RR^d$. Define $(\alpha^{(n)})_{n\in\NN}$ to be a sequence such that $\alpha^{(n)}$ converges to $x$, and if $\alpha_i^{(n)} > \alpha_j^{(n)}$, then $\alpha_i^{(m)} > \alpha_j^{(m)}$ for all $m \in \NN$. By Theorem \ref{nChoose2Case}, there exists a unique $\pi \in G$ such that for every $n\in \NN$, it follows that $\pi \cdot \alpha^{(n)} \in \mathcal{C}(P)^\circ$. Since $\pi$ is continuous, we conclude that $\pi \cdot x \in \mathcal{C}(P).$
\end{proof}

Consequently, we can write explicitly the inequality description of the interior for $\mathcal{C}(P)$:

\begin{thm}\label{kEqualsTwoIsLexMax}
The interior poset $P$ determines the interior of the lex--maximal fundamental domain. In particular, \[ \mathcal{C}(P)^\circ =  \bigcap_{\substack{i,j \in P  \\ i \gtrdot j}} \{ x\in\RR^d : x_i > x_j \} \] defines the interior of a fundamental region under the action of $G$.
\end{thm}

\begin{proof}
In order to prove the claim, we will argue that \[\mathcal{C}(P)^\circ = \mathcal{F}_{\max}^\circ.\]

Let $x \in \mathcal{C}(P)^\circ$. Since $\mathcal{C}(P)^\circ$ has the $G$--uniqueness property, for all $\pi \in G\backslash\{e\}$ we have that $\pi \cdot x \notin \mathcal{C}(P)^\circ$, meaning that at least one of the defining inequalities of $\mathcal{C}(P)^\circ$ is violated. By construction, $x = (x_{i_1}, x_{i_2}, \dots , x_{i_d})$ in $\RR^d$ is such that $x_{i_k} > x_{i_k^\prime}$ if $i_k \succ i_{k^\prime}$. Therefore, if $\pi \cdot x \notin \mathcal{C}(P)^\circ$ for all $\pi \in G\backslash \{e\}$, then $x$ is lexicographically maximal in its $G$--orbit. Hence, $x \in \mathcal{F}_{\max}$. Since $\mathcal{C}(P)^\circ$ is open, it follows that $\mathcal{C}(P)^\circ \subseteq \mathcal{F}_{\max}^\circ$.

To argue that $\mathcal{F}_{\max}^\circ \subseteq \mathcal{C}(P)^\circ$, notice firstly that $\mathcal{F}_{\max}^\circ \subseteq \mathcal{C}(P)$ because if $x \notin \mathcal{C}(P)$, then $x$ violates at least one of the relations in $P$. Then, by Proposition \ref{representativesInClosure}, let $\pi \in G$ be such that $\pi \cdot x \in \mathcal{C}(P)$. In particular, $\pi \cdot x > x$, contradicting that $x$ is lexicographically largest in its $G$--orbit.

Since $\mathcal{C}(P) = \mathcal{C}(P)^\circ \sqcup \partial\mathcal{C}(P)$, it suffices to show that if $x \in \mathcal{F}_{\max}^\circ$, then $x\notin\partial\mathcal{C}(P)$. Suppose that $x \in \mathcal{F}_{\max}^\circ$, and let $N \subseteq \mathcal{F}_{\max}$ be an open set containing $x$. We will prove by contradiction that $x \notin \partial\mathcal{C}(P)$.

Assume that $x \in \partial\mathcal{C}(P)$, and let $y \in N$ be such that $y \notin \mathcal{C}(P)$. However, $y\notin\mathcal{C}(P)$ contradicts $\mathcal{F}_{\max}^\circ \subseteq \mathcal{C}(P)$. Therefore, $\mathcal{F}_{\max}^\circ \subseteq \mathcal{C}(P)^\circ$, and we conclude that $\mathcal{F}_{\max}^\circ = \mathcal{C}(P)^\circ$.
\end{proof} 



\section{Closing Remarks}\label{closing}

Here are some questions and remarks for further research:

\begin{itemize}
\item Notice that since $S_{\binom{X}{2}} \cong S_{\binom{X}{v - 2}}$, it follows that the lex--maximal fundamental region $\mathcal{F}_{\max}$ we constructed when $k = 2$ will also work when $k = v - 2$. As a result, we can consider $t$-designs with parameters $0 < t < v - 2 < v$, where $v \ge 4$. In particular, one may be able to formulate a combinatorial reciprocity theorem for a restricted class of BIBDs.

\item Methods used in the proof of Theorem \ref{stronglyPure} could be employed to shed further light on the combinatorial structure of $\mathcal{F}_{\max}$, such as determining whether it is partitionable or shellable. (These two latter notions would need to be extended to fit our framework of partial polyhedral complexes.) Structural results of this kind  could yield a reciprocity theorem for the function counting the number of isomorphism types of $t$--designs.

\item What is an inequality description for the boundary of $\mathcal{F}_{\max}$?
\end{itemize}


The proof of Theorem \ref{nChoose2Case} gives an algorithm for transforming a point $x$ in a full--dimensional face of the braid arrangement into $x^\prime$, a representative in the same $G$--orbit that is contained in the lex--maximal fundamental cell $\mathcal{F}_{\max}$. This algorithm runs in polynomial time, so it can be used to verify whether two points represent isomorphic $t$--designs.

Observe that if we had such an algorithm for all points in $\RR^d$, then this algorithm would solve the graph isomorphism problem because $2$--dimensional faces of the braid arrangement have points that correspond to graphs. However, since the graph isomorphism problem is not expected to be polynomial--time computable, an inequality description of $\mathcal{F}_{\max}$ will be significantly more complex than our description of its interior in terms of order cones of posets.


\nocite{*}
\bibliographystyle{abbrvnat}
% use the following instead if you encounter problems
%\bibliographystyle{alpha}
\bibliography{thesisReferences}
\label{sec:biblio}

\end{document}
