First, we introduce a poset that will enable us to exhibit an inequality description for the interior of the lex--maximal fundamental cell $\mathcal{F}_{\max}$. 

Before moving further, we introduce the following group: Let \(X = \{1, 2, \dots, v\}\), where \(v > 2\), and let \(\binom{X}{2}\) be a totally ordered set on the collection of 2-subsets of \(X\). Define a group action on \(\binom{X}{2}\) by \(\mathfrak{S}_v\) via \[\pi \cdot b := \{\pi(i) : i \in b\},\] where \(\pi \in \mathfrak{S}_v\). Note such a group action will induce a group \(G \cong \mathfrak{S}_v\). 

\begin{example}
Let \(X = \{1, 2, 3, 4\}\) and 
\[
\binom{X}{2} = \left(12, 13, 14, 23, 24, 34\right),
\]
where \(ij\) is shorthand notation for \(\{i,j\}\). Then for \(\pi := (1\;2) \in \mathfrak{S}_4\),
\begin{eqnarray*}
\pi \cdot \binom{X}{2} &= &\left(\pi\cdot12, \pi\cdot13, \pi\cdot14, \pi\cdot23, \pi\cdot24, \pi\cdot34\right)\\
&= &\left(12, 23, 24, 13, 14, 34\right).
\end{eqnarray*}
Therefore, \((1\;2)\) induces the permutation \((2\;4)(3\;5) \in G\).
\end{example}

Now that we have introduced the group whose lex-maximal fundamental we propose to analyze, we define the following poset:

\begin{defn}\label{interiorPoset}
Let $X = \{1, 2, \dots , v\}$, and $d = \binom{v}{2}$. Let $A_1$ be the antichain consisting of all $2$--subsets of $X$ that do not contain $1$ nor $2$; $A_2$ be the antichain consisting of all $2$--subsets of $X$ that contain $2$, but do not contain $1$. The collection $B$ consists of those $2$--subsets containing a $1$ and it is ordered by the reverse lexicographical order. Define $P$ to be the poset on $2$--subsets of $X$ with the following relations:

\begin{itemize}
\item For any $Z \in \binom{X}{2}$, we have $\{1,2\} \succeq Z$;
\item For all $Y \in A_1$, the relation $\{1,3\} \succ Y$ holds;
\item For $\{1,i\}, \{1,j\} \in B$, we have that $\{1,i\} \succ \{1,j\}$ when $i < j$.
\end{itemize}

We call $P$ the \textbf{interior poset}. (Refer to Figure \ref{Hasse2}.)
\end{defn}

\begin{figure}[h!]
\begin{center}
\includegraphics[trim = 0mm 190mm 0mm 30mm, scale = .5]{HasseDiagramNchoose2}
\caption{Hasse diagram of $P$ in Definition \ref{interiorPoset}.}\label{Hasse2}
\end{center}
\end{figure}

For the remainder of the paper, we will reserve the letter $P$ to refer to an interior poset.

Here is a notion that will link interior posets with fundamental domains:
\begin{defn}[\cite{orderCone}]
Suppose $P$ is an interior poset with $d$ elements. The \textbf{order cone} of $P$, denoted $\mathcal{C}(P)$, is the set \[\mathcal{C}(P) \equiv \bigcap_{\substack{i,j \in P  \\ i \gtrdot j}} \{ x\in\RR^P : x_i \ge x_j \},\] where $i \gtrdot j$ means that $i$ covers $j$. Equivalently, $\mathcal{C}(P)$ is the set of all order-preserving maps $f : P \rightarrow \RR$, where we identify every possible $f$ with points $x = (x_1, x_2, \dots , x_d)$ in $\RR^P$ such that $x_i > x_j$ if and only if $f(i) > f(j)$.
\end{defn}

The next two results are intended to show why working with $P$ is helpful in constructing a fundamental cell. First, we introduce a definition.

\begin{defn}
Let $P$ be an interior poset and $\pi \in S_X$. We say that $\pi$ is \textbf{order-preserving with respect to} $\boldsymbol P$ if for every $\alpha, \beta \in P$ with $\alpha \succ \beta$, we have that either $\pi \cdot \alpha \succ \pi \cdot \beta$, or $\{\pi \cdot \alpha, \pi \cdot \beta\}$ is an antichain.
\end{defn}

\iffalse
\begin{defn}
Let $\pi \in S_X$ and $\langle \pi \rangle$ be the cyclic group generated by $\pi$. We say that $\pi$ is \textbf{free with respect to} $\boldsymbol P$ if for every $i \in P$, the orbit $\langle \pi \rangle_i$ is an antichain.
\end{defn}
\fi

\begin{prop}\label{noOrderPreservingPerms}
Let $P$ be an interior poset. Then $S_X\setminus\{e\}$ has no order-preserving permutations with respect to $P$.
\end{prop}

\begin{proof}
Suppose $\pi(1) = 2$ and $\pi(2) = 1$, or $\pi(1) = 1$ and $\pi(2) = 2$. (Otherwise, $\pi \cdot \{1, 2\} \neq \{1, 2\}$, so $\pi$ would not be order-preserving.)

\begin{enumerate}
\item Note that $\pi \cdot \{1, 3\} = \{2, \pi(3)\}$ and $\pi \cdot \{2, i\} = \{1, \pi(i)\}$. Since $\pi(3) \neq 1$, then $\pi\cdot\{1, 3\} \in B$. There exists $i \in [v]$ such that $\pi(i) = 3$, then $\pi \cdot \{2, i\} \succ \pi \cdot \{1, 3\}$, which is a contradiction.
\item Let $i > j$ such that $\pi(i) < \pi(j)$. Although $\{1, i\} \prec \{1, j\}$, we have that $\pi \cdot \{1, i\} \not \prec \pi \cdot \{1, j\}$.
\end{enumerate}

Thus, no $\pi$ in $S_X\setminus\{e\}$ is order-preserving.
\end{proof}

\begin{prop}\label{noOrderPreservingIsGUniqueness}
For every $\pi \in S_X\setminus\{e\}$, $\pi$ is not order-preserving with respect to $P$ if and only if $\mathcal{C}(P)^\circ$ has the $G$-uniqueness property.
\end{prop}

\begin{proof}
\textit{Forward implication:} Let $\pi \in S_X$ be a permutation that does not preserve the order in $P$, and let $x\in\mathcal{C}(P)$. Let $i,j\in P$ be such that $i \succ j$, and $\pi\cdot i \succ \pi\cdot j$. If $i\succ j$, then $x_i > x_j$, so for $\pi\cdot x$, we have that $x_{\pi^{-1}(i)} < x_{\pi^{-1}(j)}$. Thus, $\pi\cdot x$ is not in $\mathcal{C}(P)$.


\textit{Backward implication:} We prove the contrapositive. Take the longest chain in $P$, and take $\alpha \gtrdot \beta$ such that $\alpha$ and $\beta$ are mapped non-trivially by $\pi \in G$. Since $\pi$ is order-preserving, we have that $\pi\cdot\{\alpha, \beta\}$ is an antichain in $P$, or $\pi\cdot\alpha \succ \pi\cdot\beta$. Hence, if $x \in \mathcal{C}(P)$ satisfies $x_\alpha > x_\beta$, then $x_{\pi^{-1}(\alpha)} > x_{\pi^{-1}(\beta)}$ will be a valid inequality in $\mathcal{C}(P)$, so $\pi\cdot x\in\mathcal{C}(P)$.
\end{proof}


\iffalse
\begin{prop}
If $\pi \in S_X\setminus\{e\}$ is a free permutation, then $\pi$ is order-preserving.
\end{prop}

\begin{proof}
Let $\pi \in S_X\setminus\{e\}$ be free, and let $\alpha, \beta \in P$ be given, with $\alpha \succ \beta$. Assume that $\alpha \neq \{1,2\}$ ---otherwise, $\alpha$ and $\pi\cdot\alpha$ would be comparable, contradicting $\pi$ being a free permutation. Further assume that $\pi$ acts trivially on $\{1,2\}$, for $\alpha$ and $\pi\cdot\alpha$ would be comparable otherwise. Additionally, suppose $\pi$ acts non-trivially on $\alpha$ and $\beta$. (We will deal shortly with the case where $\pi$ acts trivially on either $\alpha$ or $\beta$.)

We will prove that either $\pi\cdot\alpha \succ \pi\cdot\beta$, or $\pi\cdot\alpha$ and $\pi\cdot\beta$ are incomparable. Since $\alpha$ and $\beta$ are comparable, it follows that $\alpha = \{1, j\}$ for some $j\in[d]$. Further, $\pi\cdot\alpha\in A_1$, or $\pi\cdot\alpha\in A_2$ because $\langle \pi \rangle_\alpha$ is an antichain.

\begin{enumerate}
\item If $\pi\cdot\alpha\in A_1$, then $\pi\cdot\beta$ is incomparable with $\pi\cdot\alpha$ because $\pi\cdot\beta \neq \{1,2\}$.
\item If $\pi\cdot\alpha\in A_2$, then either $\pi\cdot\beta\in A_1$ or $\pi\cdot\beta\prec\{1,3\}$. It is easy to see that $\pi\cdot\beta \neq \{1,3\}$: It would follow that $\beta$ and $\pi\cdot\beta$ are comparable, contradicting the freeness of $\pi$.

    Suppose $\pi\cdot\beta\notin A_1$, which implies that $\pi\cdot\beta \in B$, with $\pi\cdot\beta \prec \{1,3\}$, or $\pi\cdot\beta\in A_2$. In either case, $\pi\cdot\alpha$ and $\pi\cdot\beta$ are incomparable.

    Next suppose that $\pi\cdot\beta\not\prec\{1,3\}$, so either $\pi\cdot\beta\in A_2$ or $\pi\cdot\beta\in A_1$. Once more, we see that $\pi\cdot\alpha$ and $\pi\cdot\beta$ are not comparable.
\end{enumerate}

Notice in particular that if $\pi$ acts non-trivially on $\alpha$ and $\beta$, then $\alpha\succ\beta$ requires that $\pi\cdot\alpha$ and $\pi\cdot\beta$ be incomparable.

If we assume that $\pi$ acts trivially on either $\alpha$ or $\beta$ (but not both), then, unless either $\alpha$ or $\beta$ is $\{1,3\}$, we have that $\pi\cdot\alpha$ and $\pi\cdot\beta$ will be incomparable. If $\{1,3\}$ is acted upon non-trivially, then $\pi\cdot\{1,3\} \in A_1$, in which case again $\pi\cdot\alpha$ and $\pi\cdot\beta$ are incomparable.

We conclude that $\pi$ is order-preserving.
\end{proof}
\fi

Given this evidence that studying $P$ could be helpful, we proceed to show that the covering relations in $P$ enable us to describe the interior of $\mathcal{F}_{\max}$ in Theorem \ref{kEqualsTwoIsLexMax}. Specifically, we will show that the interior of $\mathcal{C}(P)$ is the interior of the lex--maximal fundamental cell. To go about the proof of this claim, we establish a result concerning linear extensions of $P$.

\begin{thm}\label{nChoose2Case}
Let $P$ be an interior poset, and $X$ a $v$--set. Let $v, k \in \NN$, where $v < k$. Define $\mathrm{Perm}\binom{X}{k}$ to be the set of all permutations on $\binom{X}{k}$. Let \[G \backslash \backslash \mathrm{Perm}\binom{X}{k} \equiv \left\{\mathcal{O}_{\mathfrak{S}} : \mathfrak{S}\in\mathrm{Perm}\binom{X}{k}\right\},\] that is, $G\backslash\backslash\mathrm{Perm}\binom{X}{k}$ is the set of all $G$--orbits of permutations on $\binom{X}{k}$. Then there exists a unique permutation in every $G$--orbit $\mathcal{O}_{\mathfrak{S}} \in G \backslash \backslash \mathrm{Perm}\binom{X}{k}$ that is a linear extension of $P$.
\end{thm}

\begin{proof}
\textit{Existence:} Let $T = (T_1, T_2, \dots , T_d)$ be a permutation on $\binom{X}{k}$. We will construct a $\pi\in G$ such that $\pi\cdot T$ is a linear extension of $P$.

\begin{enumerate}
\item If $T_1 \neq \{1,2\}$, then let $p\in S_X$ be such that $p\cdot T_1 = \{1,2\}$.
\item Let $\omega \equiv \min\{i\in[d]\setminus\{1\} : 1\in T_i\}$ and $\tau \equiv \min\{i\in[d]\setminus\{2\} : 2 \in T_i\}$. There are two cases to consider:
    \begin{itemize}
    \item $\mathbf{\omega < \tau:}$ Then let $q = (j\;3)$, so that $q \cdot T_\omega = \{1, 3\}$, where $j\in T_\omega$.
    \item $\mathbf{\tau < \omega:}$ Then let $q = (1 \; 2)(3 \; h)$, so that $q \cdot T_\tau = \{1, 3\}$, where $h \in T_\tau$.
    \end{itemize}
Notice that in either case, the resulting permutation $q \cdot T$ has the property that $\{1,2\}$ is the minimum, and $\{1,3\} > S$, where $S\in A_2$.

\item If $\{\{1, 2\}, \{1, 3\}, \dots , \{1, v\} \}$ is a subposet of $T$, then we are done. Otherwise, we can find a permutation $r\in S_X$ such that $r\cdot i_1 = \{1,2\} \prec r \cdot i_2 \prec \dots \prec r \cdot i_v = \{1, v\}$, where $i_j \in B$.
\end{enumerate}

We conclude that $(rqp) \cdot T$ is a linear extension of $P$.

\textit{Uniqueness:} Let $L$ be a linear extension of $P$. Let $p \in S_X \setminus \{e\}$, where $e$ denotes the identity element in $S_X$. Without loss of generality, assume $p \cdot \{1, 2\} = \{1, 2\}$. (Otherwise, the resulting permutation $p\cdot L$ no longer has $\{1, 2\}$ in the first coordinate, so $p\cdot L$ cannot possibly be a linear extension of $P$.) We have two cases to consider:

\begin{enumerate}
\item Suppose $p(1) = 1$ and $p(2) = 2$. Since $L$ is a linear extension of $P$, then $\{1, 2\} \succ \{1, 3\} \succ \dots \succ \{1, v\}$ is a subposet of $L$. Since $p$ is a non-trivial permutation, $p$ cannot respect the ordering on this subposet, so $p\cdot L$ is not a linear extension of $P$.
\item Suppose $p$ has the cycle $(1 \; 2)$. Observe that $\pi \cdot \{1, 3 \} = \{2, \pi(3)\}$ and $\pi \cdot \{2, j\} = \{1, 3\}$, for some $j\in\{2, 3, \dots, v-1\}$. Since $\{1,3\} \succ \{2, j\}$ is a covering relation in $P$, it follows that \[\pi \cdot \{1, 3\} = \{2, \pi(3)\} \not\succ \pi\cdot \{2, j\} = \{1, 3\}\] because $\pi(3) \neq 1$.
\end{enumerate}

We conclude that for any $\pi\in S_X \setminus\{e\}$, we have that $\pi \cdot L$ is not a linear extension of $P$.
\end{proof}

Remark: Let $Y = \mathrm{Perm}\binom{X}{2}$. Notice that by Burnside's lemma \cite{dummit}, \[\vert G\backslash\backslash Y\vert = \frac{1}{G}\sum_{\pi \in G} \vert \mathrm{Fix}(\pi) \vert = \frac{\binom{v}{2}!}{v!},\] where $\mathrm{Fix}(\pi)$ denotes the set of elements in $Y$ fixed by $\pi$. Therefore, $[S_d : G]$ is the number of linear extensions of $P$.

Now we can prove that $\mathcal{C}(P)$ contains all the points we need.

\begin{prop}\label{representativesInClosure}
The order cone $\mathcal{C}(P)$ has the $G$-existence property.
\end{prop}

\begin{proof}
Let $x$ be a point in $\RR^d$. Define $(\alpha^{(n)})_{n\in\NN}$ to be a sequence such that $\alpha^{(n)}$ converges to $x$, and if $\alpha_i^{(n)} > \alpha_j^{(n)}$, then $\alpha_i^{(m)} > \alpha_j^{(m)}$ for all $m \in \NN$. By Theorem \ref{nChoose2Case}, there exists a unique $\pi \in G$ such that for every $n\in \NN$, it follows that $\pi \cdot \alpha^{(n)} \in \mathcal{C}(P)^\circ$. Since $\pi$ is continuous, we conclude that $\pi \cdot x \in \mathcal{C}(P).$
\end{proof}

Consequently, we can write explicitly the inequality description of the interior for $\mathcal{C}(P)$:

\begin{thm}\label{kEqualsTwoIsLexMax}
The interior poset $P$ determines the interior of the lex--maximal fundamental domain. In particular, \[ \mathcal{C}(P)^\circ =  \bigcap_{\substack{i,j \in P  \\ i \gtrdot j}} \{ x\in\RR^d : x_i > x_j \} \] defines the interior of a fundamental region under the action of $G$.
\end{thm}

\begin{proof}
In order to prove the claim, we will argue that \[\mathcal{C}(P)^\circ = \mathcal{F}_{\max}^\circ.\]

Let $x \in \mathcal{C}(P)^\circ$. Since $\mathcal{C}(P)^\circ$ has the $G$--uniqueness property, for all $\pi \in G\backslash\{e\}$ we have that $\pi \cdot x \notin \mathcal{C}(P)^\circ$, meaning that at least one of the defining inequalities of $\mathcal{C}(P)^\circ$ is violated. By construction, $x = (x_{i_1}, x_{i_2}, \dots , x_{i_d})$ in $\RR^d$ is such that $x_{i_k} > x_{i_k^\prime}$ if $i_k \succ i_{k^\prime}$. Therefore, if $\pi \cdot x \notin \mathcal{C}(P)^\circ$ for all $\pi \in G\backslash \{e\}$, then $x$ is lexicographically maximal in its $G$--orbit. Hence, $x \in \mathcal{F}_{\max}$. Since $\mathcal{C}(P)^\circ$ is open, it follows that $\mathcal{C}(P)^\circ \subseteq \mathcal{F}_{\max}^\circ$.

To argue that $\mathcal{F}_{\max}^\circ \subseteq \mathcal{C}(P)^\circ$, notice firstly that $\mathcal{F}_{\max}^\circ \subseteq \mathcal{C}(P)$ because if $x \notin \mathcal{C}(P)$, then $x$ violates at least one of the relations in $P$. Then, by Proposition \ref{representativesInClosure}, let $\pi \in G$ be such that $\pi \cdot x \in \mathcal{C}(P)$. In particular, $\pi \cdot x > x$, contradicting that $x$ is lexicographically largest in its $G$--orbit.

Since $\mathcal{C}(P) = \mathcal{C}(P)^\circ \sqcup \partial\mathcal{C}(P)$, it suffices to show that if $x \in \mathcal{F}_{\max}^\circ$, then $x\notin\partial\mathcal{C}(P)$. Suppose that $x \in \mathcal{F}_{\max}^\circ$, and let $N \subseteq \mathcal{F}_{\max}$ be an open set containing $x$. We will prove by contradiction that $x \notin \partial\mathcal{C}(P)$.

Assume that $x \in \partial\mathcal{C}(P)$, and let $y \in N$ be such that $y \notin \mathcal{C}(P)$. However, $y\notin\mathcal{C}(P)$ contradicts $\mathcal{F}_{\max}^\circ \subseteq \mathcal{C}(P)$. Therefore, $\mathcal{F}_{\max}^\circ \subseteq \mathcal{C}(P)^\circ$, and we conclude that $\mathcal{F}_{\max}^\circ = \mathcal{C}(P)^\circ$.
\end{proof} 