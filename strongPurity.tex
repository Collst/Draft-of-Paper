We will now present one of the main results of the manuscript. Before doing so, we need the following notion.

\begin{defn}
We say that a $d$--dimensional partial polyhedral complex $\mathcal{P}$ in $\RR^d$ is \textbf{strongly pure} if for any $k$-dimensional face $\sigma$ of $\mathcal{P}$, where $0 \le k < d$, we have that $\sigma$ is a face of a $(k + 1)$-dimensional face of $\mathcal{P}$.
\end{defn}

While the notions of purity and strong purity are the same in polyhedra, they need not agree for partial polyhedral complexes. In Example \ref{nonStronglyPureExample}, for instance, we have that $\mathcal{P}$ is pure because the vertex (the only non--maximal face in $\mathcal{P}$) is a face of the relative interior of the triangle. However, $\mathcal{P}$ is not strongly pure because a vertex is a $0$--dimensional face of $T$, whereas the relative interior of $T$ is $2$--dimensional.

\begin{thm}\label{stronglyPure}
Let $G$ be a subgroup of \(\mathfrak{S}_v\) acting on $\RR^d$ by permuting coordinates. Then the boundary of $\mathcal{F}_{\max}$ is strongly pure.
\end{thm}

\begin{proof}
Before providing the proof for Theorem \ref{stronglyPure}, we give an outline. We will pick an arbitrary point $x$ in a $k$-dimensional face $\sigma$ of $\mathcal{F}_{\max}$, where $1 \le k < d$. Since the dimension of $\sigma$ is less than $d$, we know that at least one pair of coordinates of $x$ must be equal. Among all the possible coordinates that are equal, we choose a suitable index $I$ and add a small enough number $\epsilon_n(I)$, where $\epsilon_n(I) \rightarrow 0$ as $n \rightarrow \infty$. We will show by contradiction that the terms of the resulting sequence $x^{(n)}$ are lexicographically maximal in their $G$--orbit. Since the $x^{(n)}$ are in a $(k + 1)$--dimensional face $\sigma^\prime$, and $x^{(n)}$ converges to $x$, it follows that $\sigma^\prime$ has $\sigma$ as a face.

Let $\sigma$ be a $k$-dimensional face of $\mathcal{F}_{\max}$, where $0 \le k < d$, and $x = (x_1, x_2, \dots , x_d)$ be in the relative interior of $\sigma$.

We introduce a multiset consisting of coordinates of $x$. Specifically, define \[K \equiv \bigcup_{\substack{i,j \in P  \\ i \neq j}} \{x_i : x_i = x_j\}.\] We know that $K \neq \emptyset$ because $\sigma$ is a non-trivial face of $\mathcal{F}_{\max}$ ---so there is at least one pair of coordinates of $x$ that are equal.

Let $I \equiv \min\{i \in [d] : x_i = \max K\}$. Define $r \equiv \min S$, where \[S \equiv \{\vert x_i - x_j \vert : i \neq j; x_i \neq x_j\},\] and let $N\in\NN$ be such that $\frac{d}{N} < r$ when $S \neq \emptyset$. (If $S = \emptyset$, then all the coordinates of $x$ are equal. Therefore, the sequence $x^{(n)} = (x_1 + \frac{1}{n}, x_2, \dots , x_d)$ has all the desired properties. First, it is clear that $x^{(n)}$ is lexicographically maximal in its $G$-orbit. For every $n\in\NN$, we have that $x^{(n)}$ is a point in a $(k+1)$-dimensional face. Further, $x^{(n)}$ converges to $x$ as $n \rightarrow \infty$.)

We claim that for every $n\in\NN$, the sequence \[x^{(n)} = (x_1 + \epsilon_n(1), x_2 + \epsilon_n(2), \dots , x_d + \epsilon_n(d)),\] where $\epsilon_n(i) = \frac{i}{N + n}$ if $i = I$, and $\epsilon_n(i) = 0$ otherwise, is lexicographically maximal in its $G$-orbit. Observe that $(x^{(n)})_i = x_i$ as long as $i \neq I$. If $i = I$, then $(x^{(n)})_i > x_i$. As an additional observation, note for any $n\in\NN$, we have that $(x^{(n)})_I \neq (x^{(n)})_i$ for any $i \neq I$.

We will prove by contradiction that $x^{(n)} \in \mathcal{F}_{\max}$. Let $\pi \in G$ and $n\in\NN$ be such that $x^{(n)} < \pi \cdot x^{(n)}$. Let $\alpha \in [d]$ be the minimal index with the property that $(x^{(n)})_\alpha < (x^{(n)})_{\pi^{-1}(\alpha)}$. Note that if $i$ is such that $1 \le i < \alpha$, then $(x^{(n)})_i = (x^{(n)})_{\pi^{-1}(i)}$. We argue that this setup implies that $x < \pi \cdot x$.

We have two cases:

\begin{enumerate}
\item Suppose that $\alpha = I$. Then \[(x^{(n)})_I < (x^{(n)})_{\pi^{-1}(I)} \; \Longrightarrow \; x_I + \frac{I}{N + n} < x_{\pi^{-1}(I)}.\] To see why the second inequality is true, note that if $\alpha$ were a fixed point of $\pi^{-1}$, then $(x^{(n)})_\alpha < (x^{(n)})_{\pi^{-1}(\alpha)}$, contradicting $(x^{(n)})_\alpha < (x^{(n)})_{\pi^{-1}(\alpha)}$. Thus, $\pi^{-1}(I) \neq I$, implying that $x_I + \frac{I}{N + n} < x_{\pi^{-1}(I)}$.  From this last inequality, we have that $x_I < x_{\pi^{-1}(I)}$ because if $x_I = x_{\pi^{-1}(I)}$, then it follows that $x_I + \frac{I}{N + n}$ is greater than $x_{\pi^{-1}(I)}$, which is inconsistent with the inequality $(x^{(n)})_I < (x^{(n)})_{\pi^{-1}(I)}$.

    On the other hand, if there is $i\in[d]$ with $1 \le i < I$, then the equality $x_i = (x^{(n)})_i$ holds because $i\neq I$. Furthermore, since $i < \alpha$, it follows that $(x^{(n)})_i = (x^{(n)})_{\pi^{-1}(i)}$. To summarize, \[x_i = (x^{(n)})_i = (x^{(n)})_{\pi^{-1}(i)} = x_{\pi^{-1}(i)}.\]

    Hence, $x < \pi \cdot x$.
\item Now assume that $\alpha \neq I$. Note that we only need to consider the case when $\alpha < I$ because otherwise,  \[(x^{(n)})_I = (x^{(n)})_{\pi^{-1}(I)} \; \Longrightarrow \; x_{\pi^{-1}(I)} = x_I + \frac{I}{N + n},\] which is impossible due to how we chose $N$.

    Since $\alpha \neq I$, we know that $(x^{(n)})_\alpha = x_\alpha$, so $x_\alpha < (x^{(n)})_{\pi^{-1}(\alpha)}$. Now we consider two separate subcases:
    \begin{itemize}
    \item Suppose $\pi^{-1}(\alpha) = I$. By assumption, $(x^{(n)})_\alpha  <  (x^{(n)})_{\pi^{-1}(\alpha)}$, which implies \[x_\alpha < x_I + \frac{I}{N + n} \; \Longrightarrow \; x_\alpha < x_{\pi^{-1}(\alpha)},\] where the second step follows by our definition of $N$. For any $i$ with $1 \le i < \alpha$, we have \[x_i = (x^{(n)})_i = (x^{(n)})_{\pi^{-1}(i)} = x_{\pi^{-1}(i)},\] because $\pi^{-1}(i) \neq I$. Thus, $x < \pi \cdot x$.
    \item Assume now that $\pi^{-1}(\alpha) \neq I$. Since $x_\alpha < (x^{(n)})_{\pi^{-1}(\alpha)}$, it follows that $x_\alpha < x_{\pi^{-1}(\alpha)}$. For any $i\in[d]$ with $1 \le i < \alpha$, we see that \[x_i = (x^{(n)})_i = (x^{(n)})_{\pi^{-1}(i)} = x_{\pi^{-1}(i)},\] where the last equality is obtained by observing that if $\pi^{-1}(i) = I$, then $x_i = x_I + \frac{I}{N + n}$, which is not possible. Since $x_i = x_{\pi^{-1}(i)}$ for all $i < \alpha$, we infer that $x < \pi\cdot x$.
    \end{itemize}
    Therefore, in either case we reach a contradiction.
\end{enumerate}

Since the $x^{(n)}$ are in the relative interior of the same face, say $\sigma^\prime$, we conclude that for all $n\in\NN$, we have that $x^{(n)}$ is lexicographically maximal in its $G$--orbit by Lemma \ref{braidComplex}. Thus, $\sigma$ is a face of $\sigma^\prime$, so $\mathcal{F}_{\max}$ is strongly pure.
\end{proof}